\documentclass[12pt]{beamer}


\usetheme{Warsaw} % Alternatively: Warsaw, Berkeley, Copenhagen, CambrigeUS
\useoutertheme{split} % Alternatively: miniframes, infolines, split
\useinnertheme{circles}

\definecolor{UBCblue}{RGB}{50, 50, 200} % UBC Blue (primary)

\usecolortheme[named=UBCblue]{structure} % Alternatively: structure, beaver


\usepackage[utf8]{inputenc}
\usepackage[T1]{fontenc}
\usepackage[french]{babel}

\usepackage{amsmath, amsfonts, amssymb}

\title{Rendement d'une centrale Hydroélectrique}
\subtitle{Centrale de Grand'maison}
\author{Oscar Plaisant}



\begin{document}
    \frame{\titlepage}

    \section{Calcul du rendement}

    \begin{frame}{Formules}
        \pause{}
        $E_{pp} = m\cdot g\cdot h$

        \vspace{1ex}\pause

        \hspace{1em}$E_{pp}$\,:\,énergie potentielle de position\hspace{1em} $m$\,:\,masse en \texttt{kg}\hspace{1em} \texttt{g}\,:\,intensité du champ de pesanteur en $N\cdot kg^{-1}$\hspace{1em}$h$\,:\,hauteur\,en\,\texttt{m}

        \vspace{2em}\pause

        $E = P\cdot\Delta t$

        \vspace{1ex}\pause

        \hspace{1em}$E$\,:\, énergie en \texttt{J}\hspace{1em} $P$\,:\, puissance en \texttt{W}\hspace{1em}\,$\Delta t$

    \end{frame}

    \begin{frame}{Calcul du poids}

        \pause
        $V = 132 000 000 = 1,32\cdot 10^{8} m^3$\pause

        1 $m^3$ d'eau pèse 1000 kg.\pause

        $m = 1000\cdot V = 1,32\cdot 10^{11}kg$

    \end{frame}

    \begin{frame}{Calcul de l'énergie potentielle}
        \pause

        $m = 1,32\cdot 10^{11} kg$\pause

        On sait que $g = 9.80 m\cdot s^{-1}$\pause

        On sait également que $h = 926,50m$\pause

        $$
            \begin{array}{rcl}
                E_{pp} & = & m\cdot g\cdot h\\\pause
                       & = & 1,32\cdot 10^{11} \cdot 9.8\cdot 926,50\\\pause
                       & = & 1,19\cdot 10^{15} J
            \end{array}
        $$
    \end{frame}

    \begin{frame}{Calcul de la durée}
        \pause

        Le débit de la chute d'eau est $D = 217m^3s^{-1}$
        $V = 1,32\cdot 10^{8}m^3$

        $$
            \begin{array}{rcl}
                \Delta t & = & \dfrac{V}{D}\\[2ex]\pause
                         & = & \dfrac{1,32\cdot 10^{8}}{217}\\[2ex]\pause
                         & = & 608294s
            \end{array}
        $$

    \end{frame}

    \begin{frame}{Calcul de la puissance}
        \pause

        $E = E_{pp} = 1,19\cdot 10^{18} J$\pause

        On sait que $E = P\cdot \Delta t$.\pause
        On en déduit que $P = \dfrac{E}{\Delta t}$

        $$
            \begin{array}{rcl}
                P & = & \dfrac{E}{\Delta t}\\[2ex]\pause
                  & = & \dfrac{1,19\cdot 10^{15}}{608294}\\[2ex]\pause
                  & = & 1956290872 W\\\pause
                  & = & 1956 MW
            \end{array}
        $$

    \end{frame}

    \begin{frame}{Calcul du rendement}
        \pause

        $$
            \begin{array}{rcl}
                \eta & = & \dfrac{P_{reelle\, produite}}{P_{entree}}\\[2ex]\pause
                     & = & \dfrac{1956}{1800}\\[1ex]\pause
                     & = & 0.92\\\pause
                     & = & 92\%
            \end{array}
        $$
    \end{frame}

    \section{Conclusion}

    \begin{frame}{Conclusion}
        \textbf{Rendement en fonction du type de centrale}

        \begin{center}
            \includegraphics[width=1\linewidth]{graphique.png}
        \end{center}


    \end{frame}

        
\end{document}

